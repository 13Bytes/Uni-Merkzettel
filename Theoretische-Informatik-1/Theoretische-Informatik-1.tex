%%%%%%%%%%%%%%%%%%%%%%%%%%%%%%%%%%%%%%%%%%%%%%%%%%%%%%%%%%%%%%%%%%%%%%
% Template Source: Dave Richeson (divisbyzero.com), Dickinson College
% Author: Louis Dod (13bytes.de)
%%%%%%%%%%%%%%%%%%%%%%%%%%%%%%%%%%%%%%%%%%%%%%%%%%%%%%%%%%%%%%%%%%%%%%
% Please report any errors either via pull request, issue (https://github.com/13Bytes/Uni-Merkzettel) or mail (coding@13bytes.de)
% Improvements are also gladly accepted
%%%%%%%%%%%%%%%%%%%%%%%%%%%%%%%%%%%%%%%%%%%%%%%%%%%%%%%%%%%%%%%%%%%%%%


\documentclass[a4paper,10pt,landscape]{article}
\usepackage{fontspec}
\usepackage[T1]{fontenc}
\usepackage{lmodern}    % font
\usepackage[frenchb]{babel} 
\usepackage{amssymb,amsmath,amsthm,amsfonts}
\usepackage{multicol,multirow}
\usepackage{calc}
\usepackage{ifthen}
\usepackage[landscape]{geometry}
\usepackage[colorlinks=true,citecolor=blue,linkcolor=blue]{hyperref}
\usepackage[colorinlistoftodos, ngerman]{todonotes}
\usepackage{cancel}
\definecolor{lime}{RGB}{51, 204, 51}
\definecolor{neptune}{RGB}{131,194,188}

\ifthenelse{\lengthtest { \paperwidth = 11in}}
    { \geometry{top=.5in,left=.5in,right=.5in,bottom=.5in} }
	{\ifthenelse{ \lengthtest{ \paperwidth = 297mm}}
		{\geometry{top=.2cm,left=1cm,right=1cm,bottom=.2cm} }
		{\geometry{top=.2cm,left=1cm,right=1cm,bottom=.2cm} }
	}
\pagestyle{empty}
\makeatletter
\renewcommand{\section}{\@startsection{section}{1}{0mm}%
                                {-1ex plus -.5ex minus -.2ex}%
                                {0.5ex plus .2ex}%x
                                {\normalfont\large\bfseries}}
\renewcommand{\subsection}{\@startsection{subsection}{2}{0mm}%
                                {-1explus -.5ex minus -.2ex}%
                                {0.5ex plus .2ex}%
                                {\normalfont\normalsize\bfseries}}
\renewcommand{\subsubsection}{\@startsection{subsubsection}{3}{0mm}%
                                {-1ex plus -.5ex minus -.2ex}%
                                {1ex plus .2ex}%
                                {\normalfont\small\bfseries}}
                                
\newcommand{\OK}{\fcolorbox{black}{green}{\rule{0pt}{1pt}\rule{1pt}{0pt}}}
\newcommand{\NO}{\fcolorbox{black}{red}{\rule{0pt}{1pt}\rule{1pt}{0pt}}}
                                
\makeatother
\setcounter{secnumdepth}{0}
\setlength{\parindent}{0pt}
\setlength{\parskip}{0pt plus 0.5ex}
% -----------------------------------------------------------------------

\begin{document}

\raggedright
\footnotesize

Merkblatt Theo 1 - Matr.: \hspace{3cm} Name: \hspace{16cm}
{\tiny{\href{https://github.com/13Bytes/Uni-Merkzettel}{\textcolor{gray}{github.com/13Bytes/Uni-Merkzettel}}}}\\

\begin{multicols}{3}
\setlength{\premulticols}{1pt}
\setlength{\postmulticols}{1pt}
\setlength{\multicolsep}{1pt}
\setlength{\columnsep}{2pt}

% -----------------------------------------------------------------------

\section{\textcolor{lime}{Typ-3}: REG \textit{[regulär]}}
Erkannt durch \textbf{DEA = NEA = RegEx ($\gamma$)} (Robin \& Scott, regEx: Kleene) \\
Gleichbedeutend: $Synt(L)$ ist endlich.

Bsp: $\{a^nb^m | n,m \in \mathbb{N} \}$

Beweis $w \in REG$: regEx, Automat, Abschlusseigenschaften \\
Beweis $w \not\in REG$: Pumping-Lemma, Myhill-Nerode

\textcolor{magenta}{
$(u,v) \in P$ mit $v \in \Sigma \cup \Sigma V$
}

\hrule
\smallskip


\section{\textcolor{lime}{------\ }: DCFL \textit{[deterministisch kontextfrei]}}
Erkannt durch \textbf{DPDA} (Akzeptierung durch Endzustände)\\
Von jedem Zustand darf nur ein Übergang möglich sein:
$\forall a \in \Sigma, \forall z \in Z, \forall A \in  \Gamma: $
$|\delta(z,a,A)| + |\delta(z,\epsilon,A| \leq 1$

Bsp: $\{a^nb^n | n \in \mathbb{N} \}$, $\{w\$w^R | w \in \Sigma^* \}$



\section{\textcolor{lime}{Typ-2}: CFL \textit{[kontextfrei]}}
Erkannt durch \textbf{PDA} (keine kreuzenden Abhängigkeiten)

Bsp: $\{ww^R | w \in \Sigma^* \}$

Beweis $w \in CFL$: \textbf{CYK-Algo}, Grammatik oder PDA angeben\\
Beweis $w \not\in CFL$: \textbf{Pumping-Lemma T2}

\textcolor{red}{Chomsky-Normalform (CNF): $A \rightarrow a | AB$} \\
Greibach-Normalform: $A \rightarrow aV^* $ (mit $V^*, A$: Variablen)

\textcolor{magenta}{
$(u,v) \in P$ mit $u \in V$
}
\hrule
\smallskip


\section{\textcolor{lime}{Typ-1}: CSL \textit{[nicht-verkürzend]}}
\textbf{\textit{[kontextsensitiv]}}\\
Erkannt durch \textbf{LBA} (linear beschränkte Turing-Maschine)\\
(Satz von Kuroda)

Bsp: $\{a^nb^nc^n | n \in \mathbb{N} \}$, $\{a^nb^mn^nd^m | n,m \in \mathbb{N} \}$, $\{w w | w \in \Sigma^* \}$

Kuroda-Normalform (KNF): $A \rightarrow a | A | AB$ \&  $AB \rightarrow  CD$

\textcolor{magenta}{
$(u,v) \in P$ mit $ |u|\leq |v|$
}
\hrule
\smallskip


\section{\textcolor{lime}{------\ }: REC \textit{[entscheidbare Sprachen]}}
Erkannt durch Turing-Maschinen mit \textbf{JA/ NEIN}-Antwort
\hrule
\smallskip

\section{\textcolor{lime}{Typ-0}: R.E. \textit{[rekursiv aufzählbar]}}
Erkannt durch Turing-Maschinen mit \textbf{JA/ ?}-Antwort

Bsp: Halte-Problem

Abzählbar unendlich viele Grammatiken

\hrule
\smallskip

\section{Abschlusseigenschaften}
\begin{tabular}{l|lllll}
      & \begin{tabular}[c]{@{}l@{}}Schnitt\\ $\cap$\end{tabular} & \begin{tabular}[c]{@{}l@{}}Vereinig.\\ $\cup$\end{tabular} & Kompl. & Produkt & Stern \\ \hline
Typ-3 & \OK             & \OK               & \OK          & \OK      & \OK    \\
DCFL  & \NO             & \NO               & \OK          & \NO      & \NO    \\
Typ-2 & \NO             & \OK               & \NO          & \OK      & \OK    \\
Typ-1 & \OK             & \OK               & \OK          & \OK      & \OK    \\
Typ-0 & \OK             & \OK               & \NO          & \OK      & \OK   
\end{tabular}

\textcolor{neptune}{
    DCFL $\cap$ Typ-3 $\in$ DCFL \\
    Typ-2 $\cap$ Typ-3 $\in$ Typ-2 \\
}

Typ-3 ist auch unter Homomorphismen abgeschlossen


\section{Grammatiken \& Automaten}
\subsection{Grammatik}\label{sec.grammatik}
\textcolor{magenta}{
Allgemein: $G=(V,\Sigma,P,S)$ mit $P \subseteq (V\cup \Sigma)^+ \times (V\cup\Sigma)^*$}\\
Beweis $L$ wird von $G$ erzeugt: $L(G)\subseteq L$ und $L\subseteq L(G)$

Eine Grammatik heißt mehrdeutig, wenn es ein Wort mit mindestens zwei Syntaxbäumen/Ableitungen gibt.
\subsection{Automaten}
$T(M) = L$\\
$\hat{\delta}(z, w)$ beschreibt in welchen Zustand man kommt, wenn man das ganze Wort w liest
\subsubsection{DEA}
\textcolor{blue}{Allgemein: $M=(Z,\Sigma,\delta,z_0, E)$\\
$Z$: Zustände;
$z_0$: Startzustand $\in Z$;
$E$: akzeptierende Endzustände $ \subseteq Z$;
$\delta$: Überführungsfunktion $Z \times \Sigma \rightarrow Z$
}
... erzeugt Sprache T(M)

\subsubsection{NEA}
\textcolor{blue}{Allgemein: $M=(Z,\Sigma,\delta,S, E)$\\
$S$: Menge an Startzuständen;
$\delta$ Überführungsfunktion $\rightarrow \mathcal{P}(Z)$
}

\subsubsection{PDA}
\textcolor{blue}{
Allgemein: $M=(Z,\Sigma,\Gamma,\delta,z_0, \#)$\\
Mit Endzuständen: $M=(Z,\Sigma,\Gamma,\delta,z_0,\#,E)$\\
$\Gamma$: Kelleralphabet;
$\delta$ Zustandsübertragungsfunktion $\rightarrow \mathcal{P}_{\textrm{\textbf{e}ndlich}}(Z\times\Gamma^*)$
}
$\delta(z, a, A) \ni(z', B_1..B_k) $, 
oder $\epsilon$-Üg: 
$\delta(z, \epsilon, A) \ni(z', B_1..B_k) $
\textcolor{blue}{
Konfiguration (z, w, V)}
mit $L(G) = N(M) = \{ w\in\Sigma^* \ |\  \exists z \in Z: (z_0, w, \#) \vdash^* (z, \epsilon, \epsilon) \}$

\textcolor{lime}{
Lemma: $(z, w, V) \vdash^* (z', w', V') \Rightarrow (z, wx, VY) \vdash^* (z', w'x, V'Y)$
}

\subsubsection{DPDA}
Akzeptiert durch Endzustände; Immer max. ein Übergang möglich


\section{Diverses}
\subsection{Pumping-Lemma}\label{sec.pumpingLemma}
Sei $n$ gegeben. Wähle Wort $z\in L$ mit $|z|\geq n$ \\
\textbf{Typ-3} \\
Seien $u,v,w\in\Sigma^*$ beliebig in Zerlegung $z=uvw$ (wo gilt: $|v|\geq 1$ \& $|uv|\leq n$)\\
Beweis, dass $uv^iw \not\in L$ für ein $i \geq 0$
$\Rightarrow L $ nicht Typ-3


\textbf{Typ-2}\\
Seien $u,v,w,x,y\in\Sigma^*$ beliebig in Zerlegung  $z=uvwxy$ (wo gilt: $|vwx|\leq n$ \& $|vx|\geq 1$)\\
\textit{Meist Fallunterscheidung für $vx$ enthält ....}\\
Beweis, dass $uv^iwx^iy \not\in L$ für ein $i \geq 0$
$\Rightarrow L $ nicht Typ-2

\subsection{Äquivalenzen}
\textbf{Myhill-Nerode $R_L$}\\
$x R_r y \Longleftrightarrow [ \forall w\in \Sigma^*:xw\in L \Leftrightarrow yw\in L ] $  - hinten anhängen.

\textcolor{blue}{
Beweis Sprache $L$ nicht regulär:\\
}
Menge $M  \subseteq  \Sigma^*$ finden, mit $|M| = \infty$, für die gilt:\\
z.Z.: $\forall x,y \in M: x \neq y \Longrightarrow x\; \cancel{R_L}\; y\quad$\\
(Mit $w$ ist $xw\in L$, aber $yw \not\in L$)\\
Dann $L$ nicht regulär, da Index von M-N-Aquivalenz $ |\Sigma^*/R_L|=\infty$

\hrule
\smallskip
wird durch \textbf{Relation $R_M$ } verfeinert.\\
(Auf gegebenem Automaten definiert - Muss nicht minimal sein)\\
$x R_r y \Longleftrightarrow \hat{\delta}(z_o,x)= \hat{\delta}(z_o,y) $
Alle Wörter in selber Klasse, die im selben Automat-Zustand sind.
\\
$\Rightarrow$ $xR_My \Longrightarrow x R_L y \Longrightarrow |R_L| \leq |R_M|$
\hrule
\smallskip
\textbf{Syntaktische Kongruenz $\equiv_L$}\\
$x \equiv_L y  \Longleftrightarrow [ \forall w_1, w_2 \in \Sigma^*:w_1 x w_2\in L \Leftrightarrow w_1 y w_2\in L ] $  - auf beiden Seiten anhängen.

\subsection{Monoide}
\begin{itemize}
    \item Abgeschlossenheit 
    \item Assoziativ $\forall a,b,c\in M\colon (a*b)*c=a*(b*c)$
    \item neutrales Element $e$: $\forall a\in M\colon e*a=a*e=a $
\end{itemize}
$\rightarrow$ \textbf{syntaktisches Monoid $\textrm{Synth}(L) \quad \Sigma^*/\equiv_L $ }\\

\subsubsection{Erkennung durch Monoide}
Monoid $M$ erkennt $L$, wenn $A \subseteq M, \varphi=\Sigma^* \mapsto M$ und $L=\varphi^{-1}(A)$
(bzw. $w\in L \Leftrightarrow \varphi(w) \in A$)
mit $\varphi$ ist Homomorphismus.

Eine Sprache ist erkennbar, wenn sie von einem \textit{endlichen} Monoid erkannt wird.

\subsection{Homomorphismus}
Abbildung $\varphi$: Monoid $M \rightarrow N$ Monoid. Mit Eigenschaften:\\
$ \forall a,b\in M: f(a \circ_1 b) = f(a) \circ_2 f(b)$ und  $\varphi(\textrm{neutrE}_1) = \textrm{neutrE}_2$
\vspace{0.1cm}

Nützlich für Beweise. Z.B.: $\varphi(a)=a,\  \varphi(b)=b,\  \varphi(c)=\epsilon$
$L'=\Sigma^* \{c\}\Sigma^* \cap \varphi^{-1}(L)$
beschreibt Sprache $L'$ in Bezug auf $L$ mit zusätzlichem $c$ an spezieller Stelle (nur über reguläre Abschlüsse).


\subsection{Chomsky-Normalform}
\begin{itemize}
    \item Ringleitung entfernen (jeweils alle beteiligten Variablen zu neuer ändern)
    \item Variablen anordnen \& Kettenregeln entfernen
    \item Pseudoterminale einführen
    \item Abkürzungen einführen
\end{itemize}


\subsection{Minimierung DEA}
\begin{tabular}{|l|l|ll}
\cline{1-2}
\textbf{z1} &             &                                  &                                  \\ \cline{1-3}
\textbf{z2} &             & \multicolumn{1}{l|}{}            &                                  \\ \hline
\textbf{z3} &             & \multicolumn{1}{l|}{}            & \multicolumn{1}{l|}{}            \\ \hline
\textbf{}   & \textbf{z0} & \multicolumn{1}{l|}{\textbf{z1}} & \multicolumn{1}{l|}{\textbf{z2}} \\ \hline
\end{tabular}
\begin{itemize}
    \item Paare mit \textbf{einem} Endzustand markieren.
    \item Wiederholt alle Paare markieren, die für $\exists a \in \Sigma$ in Markierung landen.
    \item nicht markierte Zustandspaare verschmelzen.
\end{itemize}

\subsection{CYK-Algo}
\begin{tabular}{|l|l|ll}
\hline
Länge & w1   & \multicolumn{1}{l|}{w2}   & \multicolumn{1}{l|}{...} \\ \hline
1     & T1,1 & \multicolumn{1}{l|}{T2,1} & \multicolumn{1}{l|}{...} \\ \hline
2     & T2,1 & \multicolumn{1}{l|}{T2,2} &                          \\ \cline{1-3}
...   &      &                           &                          \\ \cline{1-2}
\end{tabular}
$T_{i,j} = \{A\in V | A \Rightarrow^*_G a_i ...a_{i+j-1}\}$

\newpage

\section{whrschl. Irrelevant}
Kardinalität = Mächtigkeit\\
endlich, abzählbar (bijektive Abbildung auf $\mathbb{N}$), überabzählbar

\subsection{Logik}
Kontraposition: $A\Rightarrow B \quad =\quad  \neg B \Rightarrow \neg A $\\
DeMorgan: $\neg(A \land B) \quad =\quad  (\neg A \lor \neg B) $

\subsubsection{Linksrekursion Entfernen (vgl. GNF)}
$A \rightarrow A\alpha_1 | A\alpha_2|...|\beta_1 |\beta_1| ...$\\
ersetzen durch:\\
$A \rightarrow \beta_1|\beta_2|...  \quad |\beta_1B|\beta_2B|...$\\
$B \rightarrow \alpha_1|\alpha_2|... \quad | \alpha_1B|\alpha_2B|... $

\subsubsection{Relationen}
für $m, m^{'(')}\in M$: \\
\textcolor{magenta}{
\textbf{Ordnungsrelationen}\\
Reflexivität: $m\ R\ m $ \\
Identität: $(m'\ R\ m) \land (m\ R\ m') \Rightarrow (m' = m)$\\
Transitivität: $(m'\ R\ m) \land (m\ R\ m'') \Rightarrow (m\ R\ m')$\\
}
Symmetrie: $(m'\ R\ m) \Leftrightarrow (m\ R\ m')$\\

Kongruenz: $[w_1 \equiv z_1 \textrm{ und } w_2 \equiv z_2 ] \Rightarrow w_1w_2 \equiv z_1z_2$

\subsection{Formale Sprachen und Alphabet}
Alphabet $\Sigma$: Nichtleere, endliche Menge\\
Formale Sprache: Teilmenge von $\Sigma^*$\\
Eine Typ-2 Sprache ist mehrdeutig, wenn jede Typ-2 Grammatik, die diese Sprache erzeugt, mehrdeutig ist.

\vspace{3cm}    % ugly fix to achieve formatting - can be removed, if more text is on this page

\section{Random Beweise}
\subsubsection{Beweis Pumping-Lemma T3}
$L$ beliebige T3-Sprache. $\Rightarrow \textrm{DEA } M$. $n=|Z|$.
Mit $x \in L, |x|\geq n \Rightarrow x=x_1x_2x_3...x_ny$ ($y\in\Sigma^*$). 
$Q \subseteq Z$ mit $Q=\{\hat{\delta}(z_0, x_1...x_n) \} \Rightarrow |Q|\leq n$.\\




\subsubsection{$\equiv_L$ ist Kongruenzrelation auf $(\Sigma^*, \cdot)$}
Seien $x,x',y,y' \in \Sigma^*$ mit $x\equiv_Lx'$ und $y\equiv_Ly'$.
Dann gilt für $u,u',v,v' \in \Sigma^*$: $uxv \in L \Leftrightarrow ux'v \in L$ und $uyv \in L \Leftrightarrow uy'v \in L$\\
Zu zeigen ist $xy \equiv_L x'y'$. Seien $u'',v'' \in \Sigma^*$:\\
$uxvy \in L \Leftrightarrow ux'yv \in L \Leftrightarrow ux'y'v \in L$

\subsubsection{Beweise mit Abschlusseigenschaften}
$A=\left\{  w\in\{a,b\}^*|\ |w|_a=|w|_b  \right\}$ ist nicht regulär.\\
Beweis durch Widerspruch: $A\cap L(a^*b^*)=\{a^nb^n|n\in\mathbb{N}\}$

\vspace{0.15cm}
$B=\left\{ a^kb^lc^m  |\ k=0 \lor l=m  \right\}$ ist nicht regulär.\\
Beweis durch Widerspruch: $L = C \cap L(aa^*b^*c^*)=\{a^nb^mc^m \ |\ n,m\in\mathbb{N} \land n\geq 1\}$\\
Betrachte den Homomorphismus $\varphi: \{a,b,c\}^* \rightarrow \{ b,c \}^*$,
der durch
$\varphi(a)=\epsilon,\ \varphi(b)=a,\ \varphi(c)=b$ definiert ist.
Da die Klasse REG unter Homo. abgeschlossen ist, ist auch
$\varphi(L) = \{ a^nb^n \ |\ n\in \mathbb{N} \}$ regulär. Widerspruch!


\vfill\null    % ugly fix to achieve formatting - can be removed, if more text is on this page
\columnbreak

\subsubsection{Exponentieller Blow-Up}
$L_k = \left\{ xay \ |\ x,y\in \{a,b\}^* \ \land \ |y|=k-1  \right\}$ \\
Jeder DEA hat min $2^k$ Zustände:

Bei Länge k existieren $2^k$ Wörter. Z.z.: Zwei Wörter $w, w'$ enden nie in gleichem Zustand: $\hat{\delta}(z_0, w) \neq \hat{\delta}(z_0, w')$

Sei $w=xay_1$, $w'=xby_2$
\textbf{Beweis durch Widerspruch:} Wenn eine Gleichheit existieren würde, müsste:
$\hat{\delta}(z_0, w)= \hat{\delta}(z_0, w')$ gelten.
Aber dann ist:
$\hat{\delta}(z_0, wx) = \hat{\delta}(\hat{\delta}(z_0, w), x)
    = \hat{\delta}(\hat{\delta}(z_0, w'), x) = \hat{\delta}(z_0, w'x)$\\
Widerspruch, da $ wx\in L_k $, aber $w'x\not\in L_k$

\vfill\null    % ugly fix to achieve formatting - can be removed, if more text is on this page
\columnbreak

\end{multicols}
\end{document}
